% ============================================================================
% 修士論文 - 帯電ダストダイナミクス解析のための個別要素法シミュレーションモデルの開発
% ============================================================================
% コンパイル方法: Menu > Compiler を "LaTeX" に設定し、.latexmkrc を作成すること
% ============================================================================
% 【変更点】ドキュメントクラスを jsreport (uplatex対応) に変更
\documentclass[a4paper,11pt,twoside,uplatex,dvipdfmx]{jsreport}

% ============================================================================
% パッケージ読み込み
% ============================================================================
% 【削除】LuaLaTeX用の設定(luatexjaなど)はすべて削除しました

\usepackage{amsmath,amssymb,amsthm}
\usepackage{bm}                     % 太字ベクトル
\usepackage[dvipdfmx]{graphicx}     % 【変更】ドライバオプションを追加
\usepackage{bmpsize}                % 【追加】画像サイズの自動計測(xbbエラー対策)
\usepackage{color}
\usepackage{subcaption}             % サブキャプション
\usepackage{booktabs}               % 表の罫線
\usepackage{tabularx}               % 柔軟な表
\usepackage{multirow}               % 表のセル結合
\usepackage{listings}               % ソースコード表示
\usepackage{algorithm}              % アルゴリズム
\usepackage{algorithmic}            % アルゴリズム
\usepackage{url}                    % URL表示
\usepackage[dvipdfmx]{hyperref}     % 【変更】ドライバオプションを追加
\usepackage{siunitx}                % SI単位系
\usepackage{physics}                % 物理記号

% 【注意】jsreportとtitlesecは競合することがあるため、一旦コメントアウトしています。
% デザイン調整が必要な場合はjsreportの仕様に合わせて再調整が必要です。
% \usepackage{titlesec}
% \titleformat{\chapter}[display]
%   {\normalfont\huge\bfseries}{第\thechapter 章}{20pt}{\Huge}
% \titlespacing*{\chapter}{0pt}{-30pt}{30pt}

% ページレイアウト
\usepackage[top=30mm, bottom=30mm, inner=30mm, outer=25mm]{geometry}

% ハイパーリンクのカラー設定(dvipdfmx用)
\hypersetup{
    colorlinks=true,
    linkcolor=blue,
    citecolor=blue,
    urlcolor=blue,
    setpagesize=false % dvipdfmxでページサイズがずれるのを防ぐ
}

% 定理環境
\theoremstyle{definition}
\newtheorem{definition}{定義}[chapter]
\newtheorem{theorem}{定理}[chapter]

% ソースコードの設定(日本語コメント対策として jlisting がない環境向けに調整)
\lstset{
    language=Fortran,
    basicstyle=\ttfamily\small,
    keywordstyle=\color{blue},
    commentstyle=\color{green!50!black},
    stringstyle=\color{red},
    numbers=left,
    numberstyle=\tiny,
    stepnumber=5,
    frame=single,
    breaklines=true,
    captionpos=b,
    escapechar=\%, % 日本語コメントのエスケープ用(必要に応じて)
}

% ============================================================================
% 本文開始
% ============================================================================
\begin{document}

% ----------------------------------------------------------------------------
% 表紙
% ----------------------------------------------------------------------------
\begin{titlepage}
\centering

\vspace*{2cm}

{\fontsize{24pt}{24pt}\selectfont 2025年度}\\[0.3cm]
{\fontsize{24pt}{24pt}\selectfont 修士論文}

\vspace{2.5cm}

{\LARGE 帯電ダストダイナミクス解析のための}\\[0.5cm]
{\LARGE 個別要素法シミュレーションモデルの開発}

\vspace{2.5cm}

{\fontsize{16pt}{16pt}\selectfont 神戸大学大学院システム情報学研究科}\\[0.2cm]
{\fontsize{16pt}{16pt}\selectfont システム情報学専攻}

\vspace{1.5cm}

{\Large 高橋 昇大}

\vspace{1cm}

% 指導教員・審査教員の表
{\fontsize{16pt}{16pt}\selectfont
\begin{tabular}{l@{\hspace{1.5em}}l@{\hspace{1em}}l@{\hspace{1em}}l}
\underline{指導教員}  & \underline{三宅 洋平 准教授} \\[0.8em]
                    & \underline{臼井 英之 教授} \\[0.8em]
                    & \underline{岩本 昌倫 特命助教} \\[0.8em]
                    
\underline{審査教員} & \underline{主査 臼井 英之 教授} \\[0.8em]
                     & \underline{副査 坪倉 誠 教授} \\[0.8em]
                     & \underline{副査 三宅 洋平 准教授} \\
\end{tabular}
}

\vfill

{\large 2026年2月5日}

\end{titlepage}

% ----------------------------------------------------------------------------
% 要旨
% ----------------------------------------------------------------------------
\chapter*{要旨}
\addcontentsline{toc}{chapter}{要旨}

本研究では、粉体挙動のシミュレーションに広く用いられる個別要素法(Distinct Element Method, DEM)に基づく粒子シミュレーションプログラムを開発し、その妥当性を検証した。

開発したプログラムは、Hertz-Mindlin接触力モデル、クーロン摩擦モデル、転がり摩擦モデル、および静電気力(クーロン力)モデルを実装している。時間積分には蛙飛び法(Leapfrog法)を採用し、計算効率化のためにセル法による近傍探索アルゴリズムを導入した。プログラムはFortran 90で実装し、約3000行のコードで構成される。

プログラムの妥当性検証として、減衰振動子モデル、壁面接触モデル、および斜面滑り・転がりモデルの3種類の検証計算を実施した。いずれも理論解との良好な一致を確認し、接触力計算および時間積分法の正確性を実証した。

応用例として、粒子堆積物の安息角シミュレーションを実施した。壁引き抜き法により形成された斜面の安息角を測定し、摩擦係数と安息角の関係を調査した。シミュレーション結果は実験的に知られている傾向と定性的に一致しており、開発したプログラムが粉体挙動の予測に有効であることを示した。

\vspace{1cm}
\noindent\textbf{キーワード:}個別要素法、粒子シミュレーション、Hertz-Mindlin接触力モデル、転がり摩擦モデル、静電気力モデル、安息角、数値計算

\newpage

% ----------------------------------------------------------------------------
% Abstract(英文要旨)
% ----------------------------------------------------------------------------
\chapter*{Abstract}
\addcontentsline{toc}{chapter}{Abstract}

In this study, a particle simulation program based on the Discrete Element Method (DEM), which is widely used for simulating granular material behavior, was developed and validated.

The developed program implements the Hertz-Mindlin contact force model, Coulomb friction model, rolling friction model, and electrostatic (Coulomb) force model. The leapfrog method was adopted for time integration, and a cell-based neighbor search algorithm was introduced for computational efficiency. The program was implemented in Fortran 90 and consists of approximately 3000 lines of code.

For validation of the program, three types of verification calculations were performed: a damped oscillator model, a wall contact model, and a slope sliding/rolling model. Good agreement with theoretical solutions was confirmed in all cases, demonstrating the accuracy of the contact force calculation and time integration method.

As an application example, repose angle simulation of particle deposits was performed. The repose angle of slopes formed by the wall withdrawal method was measured, and the relationship between friction coefficient and repose angle was investigated. The simulation results qualitatively agree with experimentally known trends, indicating that the developed program is effective for predicting granular material behavior.

\vspace{1cm}
\noindent\textbf{Keywords:} Discrete Element Method, Particle Simulation, Hertz-Mindlin Contact Force Model, Rolling Friction Model, Electrostatic Force Model, Angle of Repose, Numerical Calculation

\newpage

% ----------------------------------------------------------------------------
% 目次
% ----------------------------------------------------------------------------
\tableofcontents
\newpage



% ============================================================================
% 第1章:序論
% ============================================================================
\chapter{序論}
\label{chap:introduction}

\section{研究背景}
\label{sec:background}

粉体は、砂、土、穀物、医薬品、化学製品など、我々の身の回りに広く存在する物質形態である。粉体の挙動を理解し予測することは、土木工学における地盤安定性の評価、資源工学における鉱石処理、製薬・食品工学における混合・搬送プロセスの最適化など、極めて広範な産業分野において重要となる。さらに近年では、月面や小惑星探査(ISRU:現地資源利用)の進展に伴い、低重力・高真空環境下におけるレゴリス(堆積砂)のハンドリング技術の確立も喫緊の課題となっている。

しかし、粉体の挙動は非常に複雑であり、固体のように荷重を支える一方で、液体のように容器形状に応じて流動し、時には気体のように拡散するなど、環境やエネルギー状態に応じて劇的にその性質を変化させる。例えば、斜面上では「安息角」と呼ばれる特定の角度で安定するが、一度崩壊すれば雪崩のような流動現象を引き起こす。このような粒子間の接触、摩擦、衝突、そして微視的な力学的バランスの崩壊に起因する非線形な挙動を、従来の連続体力学のみで理論的に予測することは極めて困難である。

この問題に対し、数値シミュレーションによるアプローチとして個別要素法(Distinct Element Method, DEM)が有効である。1979年にCundallとStrackによって提案されて以来、DEMは粉体を構成する個々の粒子を独立した要素として扱い、ニュートンの運動方程式に基づいて粒子間の相互作用を逐次計算することで、系全体の巨視的な挙動を再現する標準的な手法として確立された。この手法は、粒子間の接触力モデル(バネ・ダッシュポットモデル等)を介して、粒子の並進・回転運動を追跡できるため、実験では計測困難な内部応力やエネルギー散逸構造の解明にも寄与する。

近年、計算機の劇的な性能向上に加え、GPU(Graphics Processing Unit)を用いた並列計算技術の発展により、数百万から数千万個オーダーの粒子を扱う大規模DEMシミュレーションが現実的な時間で可能となり、その適用範囲は飛躍的に拡大している。しかし、実用的なシミュレーションにおいては依然として課題も残されている。

特に、現実の粉体粒子が持つ「非球形性」を計算コストを抑えつつどう表現するかは重要なテーマであり、真球粒子に回転抵抗(転がり摩擦)を導入する手法や、複数の球を結合したクラスターモデルなどが盛んに研究されている。また、惑星探査の文脈においては、粒子径が小さくなるにつれて支配的となる静電気力(クーロン力)やファンデルワールス力などの付着力の影響を無視することができず、これらマルチフィジックスを考慮した高精度なモデル構築が求められている。

\section{研究目的}
\label{sec:objective}

前述のような月面環境下におけるレゴリスの挙動を解明するため、本研究室ではこれまで、プラズマ環境と電磁気力に基づいたダストの帯電・浮遊現象の解析が行われてきた。
しかし、これらの先行研究では、粒子間の衝突や接触といった力学的相互作用の考慮が不十分であり、粒子が堆積・流動する際の複雑な挙動を正確に再現するには至っていない。
特に、実際のレゴリス粒子は不規則な形状をしており、真球を仮定した従来のモデルでは、月面で見られるような高い安息角や特異な堆積構造を表現できないという課題があった。
そこで本研究では、月面環境における帯電粒子の挙動を高精度に予測可能なシミュレーションコードを独自に開発することを目的とする。
具体的には、個別要素法(DEM)を基盤とし、以下の3つの重要な機能を実装・検証することで、実用的な解析ツールの構築を目指す。
第一に、粒子形状の影響を考慮するための「転がり摩擦(Rolling Friction)モデル」の導入である。
計算負荷の高い非球形粒子を直接扱う代わりに、粒子の回転運動に対して抵抗トルクを与える転がり摩擦モデルを採用することで、
球形粒子の計算コストの低さを維持しつつ、不規則形状に起因する回転抑制効果や高い安息角を擬似的に再現可能とする。
第二に、「粒子間静電相互作用」の実装である。月面のような高真空環境では、粒子の帯電による静電気力(クーロン力)が支配的となる場合がある。
本研究では、機械的な接触力に加え、粒子間の長距離相互作用としての静電気力をモデル化し、帯電が粒子の凝集や分散に与える影響を評価可能なシステムを構築する。
第三に、「計算効率化手法」の適用とコードの信頼性検証である。数万から数百万粒子という大規模な系を現実的な時間で解析するために、
近傍粒子探索アルゴリズムとしてセル法(Cell Linked-List Method)を導入し、計算コストを粒子数$N$に対して線形オーダー($O(N)$)まで低減させる。
また、開発したコードの妥当性を担保するため、減衰振動、斜面滑り、衝突実験などの理論解が存在する基礎的な物理現象との比較検証を徹底して行い、数値計算としての精度と信頼性を確立する。
本研究は、これらの要素技術を統合した2次元DEMシミュレーションモデルを構築し、その基礎的な挙動検証を通じて、将来的な月面探査機器の設計やレゴリス・ハンドリング技術の開発に資する基盤ツールを提供することを最終的な到達点とする。



% ============================================================================
% 第2章:個別要素法の理論
% ============================================================================
\chapter{解析手法}
\label{chap:methodology}

本章では、個別要素法(DEM)を用いた粒子シミュレーションの解析手法について説明する。

\section{個別要素法(DEM)の基礎}
本研究では、粉体挙動の解析手法として、CundallとStrackによって提案された個別要素法(Distinct Element Method, DEM)を用いる\cite{cundall1979}。
DEMは、粉体集合体を多数の独立した粒子の集まりとして扱い、個々の粒子の運動方程式を微小時間刻み $\Delta t$ ごとに数値積分することで、系全体の挙動を追跡する手法である。

$i$番目の粒子(質量 $m_i$、慣性モーメント $I_i$)の並進運動および回転運動に関する支配方程式は、Newtonの運動方程式およびEulerの角運動方程式により、それぞれ以下のように記述される。

\begin{equation}
    m_i \frac{d\bm{v}_i}{dt} = \sum_{j} \bm{F}_{ij}^c + \sum_{k} \bm{F}_{ik}^e + m_i \bm{g}
    \label{eq:trans_motion}
\end{equation}

\begin{equation}
    I_i \frac{d\bm{\omega}_i}{dt} = \sum_{j} \bm{T}_{ij}
    \label{eq:rot_motion}
\end{equation}

ここで、$\bm{v}_i$ は並進速度、$\bm{\omega}_i$ は角速度、$\bm{g}$ は重力加速度である。
右辺の項はそれぞれ、粒子 $j$ との接触による接触力 $\bm{F}_{ij}^c$、粒子 $k$(または外部電場)から受ける静電気力などの遠隔相互作用力 $\bm{F}_{ik}^e$、および接触点に作用する力によるトルク $\bm{T}_{ij}$ を表す。
本研究では、粒子形状を球形と仮定し、計算の効率化を図りつつ、後述する転がり摩擦モデルを導入することで非球形粒子の挙動を模擬する。

\section{接触力モデル}
粒子間の接触力モデルには、粒子が微小に変形すると仮定し、その変形量(重なり量)に応じて反発力を計算する「ソフトスフィアモデル(Soft-sphere model)」を採用する。
ハードスフィアモデルでは、衝突時刻を事前に計算してイベント駆動型のシミュレーションを行うのに対し、ソフトスフィアモデルでは各粒子の運動方程式を個別に時間発展させる連続時間シミュレーションが可能である。
特に、安息角計測のような粒子が密に充填された状態での計算においては、多数の粒子が同時に接触する状況が頻繁に発生するため、粒子ごとに運動方程式を個別に解くソフトスフィアモデルの方が有効であると考えられる。
接触力 $\bm{F}^c$ は、法線方向成分 $\bm{F}_n$ と接線方向成分 $\bm{F}_t$ に分解して計算を行う。

\subsection{法線方向の接触力 (線形バネ・ダッシュポットモデル)}
\label{subsec:normal_force}

法線方向の接触力 $\bm{F}_n$ については、計算コストと物理的精度のバランスを考慮し、線形バネ・ダッシュポットモデル(Voigtモデル)を採用する。
粒子 $i$ と $j$ の接触点における法線方向の重なり量を $\delta_n$、相対速度の法線成分を $\bm{v}_n$ とすると、法線力 $\bm{F}_n$ は次式で表される。

\begin{equation}
    \bm{F}_n = -k_n \delta_n \bm{n} - \eta_n \bm{v}_n
    \label{eq:normal_force_linear}
\end{equation}

ここで、$k_n$ はバネ定数、$\eta_n$ は粘性減衰係数、$\bm{n}$ は接触面法線ベクトルである。
本研究では、これらのパラメータを以下の手順で決定している。

\subsubsection*{バネ定数 $k_n$ の決定}
線形モデルであっても実際の粒子の弾性特性を反映させるため、Hertzの接触理論に基づきバネ定数を決定する。
Hertz理論では、接触力 $F_H$ は重なり量 $\delta_n$ の1.5乗に比例する($F_H \propto \delta_n^{3/2}$)。
そこで、粒子の代表的な最大食い込み量(参照食い込み量)を $\delta_{ref}$ とし、この変位において線形モデルの弾性力がHertz解と一致するように $k_n$ を設定する。
本研究では、$\delta_{ref}$ を平均粒子半径の5\%とし、次式を用いて算出する。

\begin{equation}
    k_n = \frac{4}{3} E^* \sqrt{R^*} \sqrt{\delta_{ref}}
    \label{eq:kn_derivation}
\end{equation}

ここで、$E^*$ は有効ヤング率、$R^*$ は有効半径であり、それぞれの粒子($i, j$)の物性値を用いて以下のように定義される。
\begin{equation}
    \frac{1}{E^*} = \frac{1-\nu_i^2}{E_i} + \frac{1-\nu_j^2}{E_j}, \quad
    \frac{1}{R^*} = \frac{1}{R_i} + \frac{1}{R_j}
\end{equation}

\subsubsection*{減衰係数 $\eta_n$ の決定}
減衰係数 $\eta_n$ は、衝突におけるエネルギー散逸を規定する反発係数 $e$ と直接関係している。
減衰振動の運動方程式の解析解に基づき、所望の反発係数 $e$ を再現するための $\eta_n$ は、バネ定数 $k_n$ を用いて次式で与えられる。

\begin{equation}
    \eta_n = - \frac{2 \ln e}{\sqrt{\pi^2 + (\ln e)^2}} \sqrt{m^* k_n}
    \label{eq:eta_derivation}
\end{equation}

ここで、有効質量 $m^*$ は、粒子 $i$ と $j$ の質量を $m_i$ と $m_j$ とすると、次式で与えられる。

\begin{equation}
    m^* = \frac{m_i m_j}{m_i + m_j}
\end{equation}

この式を用いることで、数値シミュレーションにおいて設定した反発係数を正確に再現することが可能となる。

\subsection{接線方向の接触力}
接線方向の接触力 $\bm{F}_t$ は、接触面での滑りとスティック(固着)状態を表現するために、Mindlin-Deresiewiczの理論を簡略化したバネ・スライダーモデルを用いる。
接線方向の累積変位を $\bm{\delta}_t$ とすると、試行的な接線力 $\bm{F}_t^*$ は線形バネにより計算される。

\begin{equation}
    \bm{F}_t^* = -k_t \bm{\delta}_t
\end{equation}

ただし、Coulombの摩擦法則に基づき、接線力の上限は摩擦係数 $\mu_s$ と法線力 $|\bm{F}_n|$ の積で制限される。
したがって、最終的な接線力 $\bm{F}_t$ は以下の条件に従う。

\begin{equation}
    \bm{F}_t = 
    \begin{cases} 
    \bm{F}_t^* & (|\bm{F}_t^*| \leq \mu_s |\bm{F}_n|) \\
    \mu_s |\bm{F}_n| \frac{\bm{F}_t^*}{|\bm{F}_t^*|} & (|\bm{F}_t^*| > \mu_s |\bm{F}_n|)
    \end{cases}
    \label{eq:tangential_force}
\end{equation}

これにより、粒子間の滑り現象をモデル化している。

\subsection{転がり摩擦 (Rolling Friction)}
実際のレゴリス粒子は不規則な形状をしており、球形粒子モデルでは回転が過剰に発生し、安息角が低く見積もられる傾向がある。
この問題を解決するため、本研究では転がり摩擦(Rolling Friction)モデルを導入する。
これは、粒子の相対回転運動に対して抵抗となるトルク $\bm{T}_r$ を与えるものであり、次式で定義される。

\begin{equation}
    \bm{T}_r = -\mu_r R^* |\bm{F}_n| \frac{\bm{\omega}_{rel}}{|\bm{\omega}_{rel}|}
    \label{eq:rolling_friction}
\end{equation}

ここで、$\mu_r$ は転がり摩擦係数、$R^*$ は接触点までの距離、$\bm{\omega}_{rel}$ は相対角速度である。
このトルクを運動方程式(\ref{eq:rot_motion})に付加することで、粒子の回転を抑制し、非球形粒子特有の「噛み合わせ」効果を擬似的に表現する。

\section{粒子間静電相互作用}
\label{sec:electrostatics}

月面環境等の高真空下では、粒子の帯電による静電気力が支配的となる場合がある。
一般に、電荷 $q_i, q_j$ を持つ2粒子間には、中心間距離 $r_{ij}$ に応じて以下のクーロン力 $\bm{F}_{ij}^{Coulomb}$ が作用する。

\begin{equation}
    \bm{F}_{ij}^{Coulomb} = k_e \frac{q_i q_j}{r_{ij}^3} \bm{r}_{ij} = k_e \frac{q_i q_j}{r_{ij}^2} \bm{n}_{ij}
    \label{eq:coulomb_raw}
\end{equation}

ここで、$k_e = 1/(4\pi\varepsilon_0)$ はクーロン定数、$\bm{r}_{ij} = \bm{r}_j - \bm{r}_i$ は粒子 $i$ から $j$ への相対位置ベクトル、$\bm{n}_{ij} = \bm{r}_{ij}/r_{ij}$ は単位ベクトルである。

式(\ref{eq:coulomb_raw})をそのまま数値計算に用いる際には、以下の2つの問題が生じる:
\begin{enumerate}
    \item \textbf{近距離での発散}:$r_{ij} \to 0$ のとき力が無限大に発散し、数値不安定を引き起こす
    \item \textbf{遠距離での計算コスト}:全粒子対の相互作用を計算すると $O(N^2)$ の計算量となり、大規模計算が困難となる
\end{enumerate}

本研究では、これらの問題を解決するため、近距離正則化と遠距離カットオフの2つの手法を組み合わせて採用した。

\subsection{近距離クーロン力の正則化}

粒子間距離 $r_{ij}$ が極めて小さくなった場合の力の発散を防ぐため、ソフトニングパラメータ $\delta > 0$ を導入し、有効距離 $r_{ij}^{eff}$ を定義する:

\begin{equation}
    r_{ij}^{eff} = \sqrt{r_{ij}^2 + \delta^2}
    \label{eq:softening_distance}
\end{equation}

この有効距離を用いることで、ソフトニングを施したクーロン力は以下のように表される:

\begin{equation}
    \bm{F}_{ij}^{soft} = k_e \frac{q_i q_j}{\left(r_{ij}^2 + \delta^2\right)^{3/2}} \bm{r}_{ij}
    \label{eq:coulomb_softened}
\end{equation}

式(\ref{eq:coulomb_softened})において、$r_{ij} \to 0$ のとき力は有限値 $k_e q_i q_j \bm{r}_{ij} / \delta^3$ に収束し、発散が回避される。
一方、$r_{ij} \gg \delta$ の領域では $r_{ij}^{eff} \approx r_{ij}$ となり、本来のクーロン力(式\ref{eq:coulomb_raw})に漸近する。
本研究では、ソフトニングパラメータとして平均粒子半径の5\%($\delta = 0.05 \bar{R}$)を使用した。

\subsection{遠距離クーロン力のカットオフ}

計算コストを低減するため、カットオフ距離 $r_c$ を導入し、$r_{ij} \geq r_c$ の粒子対については相互作用を無視する。
しかし、単純に $r_{ij} = r_c$ で力を打ち切ると、その境界で力の不連続性が生じ、エネルギー保存則の悪化や数値計算上の不安定性を引き起こす。

この問題を回避するため、本研究では$r_{ij} = r_c$ において力が厳密にゼロとなるよう、定数項を差し引く補正を行った。
カットオフ境界での有効距離を $r_c^{eff} = \sqrt{r_c^2 + \delta^2}$ と定義すると、最終的な静電気力は次式で与えられる:

\begin{equation}
    \bm{F}_{ij}^{elec} = 
    \begin{cases}
    \displaystyle k_e q_i q_j \left[ \frac{1}{\left(r_{ij}^2 + \delta^2\right)^{3/2}} - \frac{1}{\left(r_c^2 + \delta^2\right)^{3/2}} \right] \bm{r}_{ij} & (r_{ij} < r_c) \\[12pt]
    \bm{0} & (r_{ij} \geq r_c)
    \end{cases}
    \label{eq:coulomb_shifted}
\end{equation}

式(\ref{eq:coulomb_shifted})の物理的意味を以下に整理する:
\begin{itemize}
    \item 第1項 $1/(r_{ij}^2 + \delta^2)^{3/2}$:ソフトニングを施した本来のクーロン力項
    \item 第2項 $1/(r_c^2 + \delta^2)^{3/2}$:$r_{ij} = r_c$ で力をゼロにするためのシフト補正項(定数)
\end{itemize}

この定式化により、$r_{ij} = r_c$ において括弧内がゼロとなり、力が連続的にゼロへ収束する。
近傍粒子探索にはセルリスト法を併用することで、計算コストを $O(N)$ に低減しつつ、物理的に矛盾のない連続的な力場を実現している。

\section{数値積分法}
運動方程式の数値積分には、蛙飛び法(Leapfrog法)を採用する。
Leapfrog法は、位置と速度を半ステップずらして定義するシンプレクティック積分法の一種であり、エネルギー保存性が高く、長期的なシミュレーションにおいて安定性が高いという特徴を持つ。
時刻 $t$ における位置 $\bm{x}(t)$ と時刻 $t-\Delta t/2$ における速度 $\bm{v}(t-\Delta t/2)$ から、次のステップの物理量を以下のように更新する。

\begin{equation}
    \bm{v}(t + \Delta t/2) = \bm{v}(t - \Delta t/2) + \bm{a}(t) \Delta t
\end{equation}

\begin{equation}
    \bm{x}(t + \Delta t) = \bm{x}(t) + \bm{v}(t + \Delta t/2) \Delta t
\end{equation}

\section{計算効率化手法}
粒子数 $N$ が増加すると、接触判定や近距離力の計算コストが $O(N^2)$ で増大する問題がある。
これを解決するため、計算領域を格子状のセルに分割し、近傍粒子探索を効率化するセル・リンクト・リスト法(Cell Linked-List Method)を導入した。
本手法では、各粒子が属するセルを特定し、接触判定の対象を「自粒子が属するセル」および「隣接するセル」内の粒子のみに限定する。
これにより、探索計算のオーダーを $O(N)$ まで低減させ、大規模なシミュレーションを現実的な時間で実行可能とした。

% ============================================================================
% 第3章:計算精度の検証
% ============================================================================
\chapter{計算精度の検証}
\label{chap:accuracy_verification}

\section{検証の概要}
本研究で開発したDEMプログラムの信頼性を担保するため、理論解が存在する基礎的な物理系を用いた検証計算を行った。
検証項目は以下の5点である:
\begin{enumerate}
    \item 法線方向の接触力モデルと時間積分法(減衰振動)
    \item 粒子-壁間衝突における Hertz 接触モデル(最大重なり量)
    \item 粒子-粒子間衝突における運動量・エネルギー保存則
    \item 接線方向摩擦および転がり摩擦モデル(斜面滑り・転がり運動)
    \item 静電気力モデル(2粒子間クーロン相互作用)
\end{enumerate}
それぞれの検証において、数値解析結果と理論解を比較し、その妥当性を評価した。

\section{法線方向の接触力の検証(減衰振動)}
\label{sec:valid_normal}

法線方向の接触力モデル(線形バネ-ダンパモデル)と時間積分法(Leapfrog法)の妥当性を検証するため、単一粒子の減衰振動シミュレーションを実施した。
質量 $m$ の粒子を壁面に接触させ、初期変位を与えて自由振動させた際の挙動を解析した。

\subsection{理論解}
系の運動方程式は、バネ定数 $k_n$、減衰係数 $\eta_n$ を用いて次式で表される。
\begin{equation}
    m \ddot{x} + \eta_n \dot{x} + k_n x = 0
    \label{eq:damped_eom}
\end{equation}
不足減衰条件($\eta_n < 2\sqrt{mk_n}$)における理論解は次式となる。
\begin{equation}
    x(t) = A e^{-\gamma t} \cos(\omega_d t + \phi)
    \label{eq:damped_solution}
\end{equation}
ここで、$\gamma = \eta_n / 2m$ は減衰率、$\omega_d = \sqrt{k_n/m - \gamma^2}$ は減衰固有角振動数である。
初期条件 $x(0) = x_0$, $\dot{x}(0) = v_0$ を満たす定数 $A$, $\phi$ は次式で与えられる:
\begin{equation}
    A = \sqrt{x_0^2 + \left(\frac{v_0 + \gamma x_0}{\omega_d}\right)^2}, \quad
    \phi = \arctan\left(-\frac{v_0 + \gamma x_0}{\omega_d x_0}\right)
\end{equation}

また、減衰係数 $\eta_n$ は目標とする反発係数 $e$ から次式で計算される:
\begin{equation}
    \eta_n = -\frac{2 \ln(e) \sqrt{m k_n}}{\sqrt{\ln^2(e) + \pi^2}}
    \label{eq:damping_from_e}
\end{equation}

\subsection{検証結果}
図\ref{fig:damped_oscillator}に、数値解析結果と理論解(式\ref{eq:damped_solution})の比較を示す。
粒子の変位および速度の時刻歴は理論解と極めて良好に一致しており、法線方向の接触力計算およびLeapfrog法による時間積分が正しく実装されていることを確認した。
さらに、数値解から算出した反発係数は設定値($e=0.8$)と相対誤差1\%以内で一致した。

\begin{figure}[htbp]
    \centering
    \includegraphics[width=0.8\textwidth]{figures/damped_oscillator.png}
    \caption{減衰振動における変位・速度の時刻歴応答比較。実線が数値解、破線が理論解を示す。}
    \label{fig:damped_oscillator}
\end{figure}

\section{衝突計算の妥当性検証}
\label{sec:valid_collision}

法線方向の接触力モデル(Hertz-Mindlin)が、衝突現象におけるエネルギー保存則や運動量保存則を満たし、理論通りの反発挙動を示すかを検証するため、1次元の衝突シミュレーションを実施した。

\subsection{粒子-壁間の衝突(最大重なり量の検証)}
\label{sec:wall1d}

質量 $m$、半径 $R$ の粒子が、初速度 $v_0$ で剛体壁に垂直に衝突する系を考える。

\subsubsection{理論解の導出}
粘性減衰を無視した弾性衝突($e=1.0$)の場合、衝突による運動エネルギーの減少分はすべて接触点における弾性ポテンシャルエネルギーに変換される。
Hertz接触理論における法線力は、重なり量(オーバーラップ)$\delta$ に対して次式で与えられる:
\begin{equation}
    F_n = k_n \delta^{3/2}
    \label{eq:hertz_force}
\end{equation}
ここで $k_n$ は Hertz 剛性であり、等価ヤング率 $E^*$ と等価半径 $R^*$ を用いて次式で定義される:
\begin{equation}
    k_n = \frac{4}{3} E^* \sqrt{R^*}
\end{equation}

式(\ref{eq:hertz_force})を積分することで、弾性ポテンシャルエネルギー $U(\delta)$ が得られる:
\begin{equation}
    U(\delta) = \int_0^\delta F_n \, d\delta' = \frac{2}{5} k_n \delta^{5/2}
\end{equation}

エネルギー保存則 $\frac{1}{2}mv_0^2 = U(\delta_{\text{max}})$ より、理論的な最大重なり量 $\delta_{\text{max}}$ は次式で導出される:
\begin{equation}
    \delta_{\text{max}} = \left( \frac{5 m v_0^2}{4 k_n} \right)^{2/5}
    \label{eq:delta_max}
\end{equation}

\subsubsection{検証結果}
シミュレーションでは、粒子($m=1.0\,\mathrm{kg}$, $R=0.1\,\mathrm{m}$)を初速度 $v_0=1.0\,\mathrm{m/s}$ で壁に衝突させた。
図\ref{fig:wall_collision}に、衝突過程における粒子の重なり量の時刻歴を示す。
数値解析によって得られた最大重なり量は、式(\ref{eq:delta_max})から算出される理論値と相対誤差0.1\%以下で一致しており、Hertz接触モデルの非線形バネ挙動が正しく計算されていることを確認した。

\begin{figure}[htbp]
    \centering
    \includegraphics[width=0.8\textwidth]{figures/overlap_validation.png}
    \caption{粒子-壁衝突における重なり量と速度の時刻歴。上段:重なり量、下段:相対速度。実線が数値解、破線が理論解を示す。}
    \label{fig:wall_collision}
\end{figure}

\subsection{粒子-粒子間の衝突(運動量・エネルギー保存則の検証)}
\label{sec:particle1d}

次に、同質量・同形状の2粒子を用いた1次元正面衝突試験を行った。

\subsubsection{理論解}
質量 $m$ の粒子1(速度 $v_1$)と粒子2(速度 $v_2$)が完全弾性衝突($e=1.0$)する場合、運動量保存則およびエネルギー保存則より:
\begin{align}
    m v_1 + m v_2 &= m v'_1 + m v'_2 \quad \text{(運動量保存)} \\
    \frac{1}{2}m v_1^2 + \frac{1}{2}m v_2^2 &= \frac{1}{2}m v'^2_1 + \frac{1}{2}m v'^2_2 \quad \text{(エネルギー保存)}
\end{align}

これらを連立して解くと、同質量の場合の衝突後速度は:
\begin{equation}
    v'_1 = v_2, \quad v'_2 = v_1
    \label{eq:elastic_collision}
\end{equation}
すなわち、速度が完全に入れ替わる。

\subsubsection{検証結果}
粒子1($v_1 = v_0$)と静止した粒子2($v_2 = 0$)を衝突させた。
図\ref{fig:particle_collision}に、衝突前後の2粒子の速度変化を示す。
衝突過程において粒子1は減速し、粒子2は加速され、最終的に速度が完全に入れ替わっていることが確認できる(式\ref{eq:elastic_collision}と一致)。
また、衝突前後での全運動量 $p = m(v_1 + v_2)$ の変化は数値誤差の範囲内(相対誤差 $10^{-10}$ 以下)でゼロであり、運動量保存則が厳密に成立していることを実証した。

\begin{figure}[htbp]
    \centering
    \includegraphics[width=0.8\textwidth]{figures/particle_collision_validation.png}
    \caption{2粒子間の正面衝突における速度変化の検証。衝突後に速度が入れ替わる弾性衝突の特性が正しく再現されている。}
    \label{fig:particle_collision}
\end{figure}

\section{接線方向および転がり摩擦の検証(斜面滑り試験)}
\label{sec:valid_slope}

摩擦モデルの妥当性を検証するため、傾斜角 $\theta$ の斜面上を滑走・転動する球形粒子(半径 $R$、質量 $m$、慣性モーメント $I = \frac{2}{5}mR^2$)の運動を解析した。

\subsection{理論的背景}

斜面上の球の運動は、接線方向摩擦係数 $\mu$ の大きさによって「滑り」と「転がり」の2つのモードに分類される。

\subsubsection{臨界摩擦係数}
球が滑らずに純粋に転がるためには、摩擦力が重力の斜面方向成分による加速トルクを支える必要がある。
並進と回転の運動方程式:
\begin{align}
    m a &= m g \sin\theta - F_f \\
    I \dot{\omega} &= F_f R
\end{align}
および純転がり条件 $a = R\dot{\omega}$ を連立すると、転がり条件を満たすための臨界摩擦係数 $\mu_c$ が導かれる:
\begin{equation}
    \mu_c = \frac{2}{7} \tan\theta
    \label{eq:critical_friction}
\end{equation}

\subsubsection{滑り条件($\mu < \mu_c$)}
摩擦係数が臨界値未満の場合、粒子は滑りながら斜面を下る。このとき:
\begin{equation}
    a_{\text{slide}} = g(\sin\theta - \mu \cos\theta)
    \label{eq:slide_accel}
\end{equation}

\subsubsection{転がり条件($\mu \geq \mu_c$)}
摩擦係数が臨界値以上の場合、粒子は滑らずに転がる。このとき:
\begin{equation}
    a_{\text{roll}} = \frac{5}{7} g \sin\theta = \frac{g \sin\theta}{1 + I/(mR^2)}
    \label{eq:roll_accel}
\end{equation}
また、純転がり条件より角速度は $\omega = v/R$ の関係を満たす。

\subsection{検証結果}

\subsubsection{滑り摩擦の検証}
傾斜角 $\theta = 45°$、摩擦係数 $\mu = 0.2$ の条件でシミュレーションを行った。
この条件では $\mu_c = \frac{2}{7}\tan(45°) \approx 0.286$ であり、$\mu < \mu_c$ なので滑り条件となる。
理論加速度は式(\ref{eq:slide_accel})より $a = g(\sin 45° - 0.2 \cos 45°) \approx 5.55\,\mathrm{m/s^2}$ と予測される。
数値解析結果は理論値と相対誤差1.0\%未満で一致した。

\subsubsection{転がり摩擦の検証}
摩擦係数 $\mu = 0.5$($> \mu_c$)の転がり条件でシミュレーションを行った。
理論加速度は式(\ref{eq:roll_accel})より $a = \frac{5}{7} g \sin 45° \approx 4.95\,\mathrm{m/s^2}$ である。
図\ref{fig:slope_velocity}に、斜面上の粒子の並進速度の時刻歴を示す。
数値解析結果は理論加速度と良好に一致し、転がり条件が正しく実現されていることを確認した。

さらに、純転がり条件では角速度 $\omega$ と並進速度 $v$ の間に $\omega = v/R$ の関係(滑りなし条件)が成り立つ。
図\ref{fig:slope_angular_velocity}に角速度の時刻歴を示す。
数値解析で得られた角速度は、速度から計算した理論値 $\omega = v/R$ と一致しており、接触点で滑りが生じていないことを確認した。

\begin{figure}[htbp]
    \centering
    \includegraphics[width=0.8\textwidth]{figures/slope_velocity.png}
    \caption{斜面滑り試験における粒子の並進速度の時刻歴。転がり条件($\mu=0.5$)において、理論加速度 $a = \frac{5}{7}g\sin\theta$ と数値解が良好に一致している。}
    \label{fig:slope_velocity}
\end{figure}

\begin{figure}[htbp]
    \centering
    \includegraphics[width=0.8\textwidth]{figures/slope_angular_velocity.png}
    \caption{転がり条件における角速度の時刻歴。数値解(実線)と理論値 $\omega = v/R$(破線)が一致しており、純転がり条件が満たされていることを確認できる。}
    \label{fig:slope_angular_velocity}
\end{figure}

\section{静電気力の検証(2粒子間相互作用)}
\label{sec:valid_coulomb}

本研究で導入した静電気力(クーロン力)の実装精度を検証するため、真空中における帯電した2粒子間の反発運動を解析した。

\subsection{問題設定}
同質量 $m$、同符号電荷 $q$ を持つ2粒子を初期距離 $r_0$ で静止させ、クーロン力のみで運動させる系を考える。
重力や接触力は無視し、1次元問題として扱う。

\subsection{理論解の導出}

\subsubsection{相対座標における運動方程式}
粒子1の位置を $x_1$、粒子2の位置を $x_2$ とし、相対距離 $r(t) = x_2 - x_1$ を導入する。
各粒子の運動方程式:
\begin{equation}
    m \ddot{x}_1 = -\frac{k_e q^2}{r^2}, \quad m \ddot{x}_2 = +\frac{k_e q^2}{r^2}
\end{equation}
を差し引くと、相対距離の運動方程式が得られる:
\begin{equation}
    \ddot{r} = \frac{2 k_e q^2}{m r^2}
    \label{eq:coulomb_relative_eom}
\end{equation}

\subsubsection{エネルギー積分}
初期条件 $r(0) = r_0$, $\dot{r}(0) = 0$(静止出発)のもとで、エネルギー保存則より:
\begin{equation}
    E = \frac{m}{4} \dot{r}^2 + \frac{k_e q^2}{r} = \frac{k_e q^2}{r_0} = \text{const.}
\end{equation}
これを解くと、相対速度は:
\begin{equation}
    \dot{r} = \sqrt{\frac{4 k_e q^2}{m} \left( \frac{1}{r_0} - \frac{1}{r} \right)}
    \label{eq:coulomb_velocity}
\end{equation}
各粒子の速度は対称性より $v = \frac{1}{2}\dot{r}$ である。

\subsubsection{時刻と距離の関係}
式(\ref{eq:coulomb_velocity})を積分すると、時刻 $t$ と相対距離 $r$ の関係が得られる:
\begin{equation}
    t(r) = \tau \left[ \sqrt{\frac{r}{r_0}} \sqrt{\frac{r}{r_0} - 1} + \ln\left( \sqrt{\frac{r}{r_0} - 1} + \sqrt{\frac{r}{r_0}} \right) \right]
    \label{eq:coulomb_time_distance}
\end{equation}
ここで、$\tau$ は特性時間スケールであり:
\begin{equation}
    \tau = \sqrt{\frac{m r_0^3}{4 k_e q^2}}
\end{equation}
と定義される。式(\ref{eq:coulomb_time_distance})の逆関数を数値的に求めることで、任意の時刻における $r(t)$ が得られる。

\subsection{検証結果}
図\ref{fig:coulomb_valid}に、数値解と理論解(式\ref{eq:coulomb_time_distance}の逆関数)の比較を示す。
上段は粒子間距離、中段は粒子速度、下段はクーロン力の大きさの時刻歴である。
カットオフ距離を十分大きく設定($r_c \gg r_0$)した場合、数値解は理論解と極めて良好に一致し、相対誤差は $10^{-6}$ 以下であった。
これにより、長距離力である静電気力が正しく計算されていることを確認した。

\begin{figure}[htbp]
    \centering
    \includegraphics[width=0.8\textwidth]{figures/coulomb_two_particles_comparison.png}
    \caption{2粒子間クーロン相互作用の検証。上段:粒子間距離、中段:粒子速度、下段:クーロン力。実線が数値解、破線が理論解を示す。}
    \label{fig:coulomb_valid}
\end{figure}

\section{本章のまとめ}
以上の検証計算により、本研究で開発したDEMコードにおける各物理モデルが理論解と良好に一致することを確認した。
表\ref{tab:validation_summary}に検証結果の概要を示す。

\begin{table}[htbp]
\centering
\caption{検証計算の概要と結果}
\label{tab:validation_summary}
\begin{tabular}{lll}
\hline
検証項目 & 検証内容 & 相対誤差 \\
\hline
減衰振動 & 時間積分法・減衰力モデル & $< 1\%$ \\
粒子-壁衝突 & Hertz接触・最大重なり量 & $< 0.1\%$ \\
粒子-粒子衝突 & 運動量・エネルギー保存 & $< 10^{-10}$ \\
斜面滑り & 滑り・転がり摩擦モデル & $< 1\%$ \\
クーロン相互作用 & 静電気力計算 & $< 10^{-6}$ \\
\hline
\end{tabular}
\end{table}

これらの結果から、本コードは物理法則に従って正しく実装されており、数値計算上の精度も十分に確保されていると判断できる。
次章以降では、この検証済みのコードを用いて、より複雑な多粒子系(月面ダストの安息角シミュレーション等)の解析を行う。


% ============================================================================
% 第4章:安息角シミュレーション
% ============================================================================
\chapter{安息角シミュレーション}
\label{chap:repose_angle}

本章では、開発したDEMプログラムを用いた安息角シミュレーションについて説明する。

\section{安息角とは}
\label{sec:repose_angle_def}

安息角(Angle of Repose)とは、粉体を堆積させたときに形成される斜面が安定を保つことのできる最大の傾斜角である。安息角は粉体の流動性を評価する重要な指標であり、粒子の形状、大きさ、表面状態、摩擦係数などに依存する。

\section{計算手法:壁引き抜き法}
\label{sec:wall_withdrawal}

本研究では、壁引き抜き法により安息角を測定した。この方法は以下の手順で行う。

\begin{enumerate}
    \item 矩形容器内に粒子を充填し、重力下で静止させる
    \item 容器の側壁を瞬間的に除去(引き抜き)する
    \item 粒子が崩落し、新たな斜面が形成される
    \item 斜面の傾斜角を安息角として測定する
\end{enumerate}

斜面の傾斜角は、表面粒子の座標データに対して線形回帰を行い、得られた傾きから算出した。

図\ref{fig:wall_withdrawal}に、壁引き抜き法によるシミュレーションの様子を示す。
(a)は壁引き抜き前の粒子充填状態、(b)は壁引き抜き後に粒子が崩落して斜面を形成した状態である。
赤線は引き抜かれる右壁を示している。

\begin{figure}[htbp]
    \centering
    \begin{subfigure}[b]{\textwidth}
        \centering
        \includegraphics[width=\textwidth]{figures/wall_before_withdrawal.png}
        \caption{壁引き抜き前:矩形容器内に充填された粒子}
    \end{subfigure}
    
    \vspace{0.5cm}
    
    \begin{subfigure}[b]{\textwidth}
        \centering
        \includegraphics[width=\textwidth]{figures/wall_after_withdrawal.png}
        \caption{壁引き抜き後:崩落により形成された斜面}
    \end{subfigure}
    \caption{壁引き抜き法による安息角計測シミュレーションの様子。上段が壁引き抜き前、下段が壁引き抜き後の粒子配置を示す。}
    \label{fig:wall_withdrawal}
\end{figure}

\section{計算条件}
\label{sec:simulation_conditions}

安息角シミュレーションに用いた計算条件を表\ref{tab:repose_params}に示す。

\begin{table}[htbp]
\centering
\caption{安息角シミュレーションの計算条件}
\label{tab:repose_params}
\begin{tabular}{lll}
\toprule
パラメータ & 値 & 単位 \\
\midrule
粒子半径 $r$ & 1.0 & mm \\
粒子密度 $\rho$ & 2480 & kg/m$^3$ \\
ヤング率(粒子)$E_p$ & $4.9 \times 10^9$ & Pa \\
ヤング率(壁)$E_w$ & $3.9 \times 10^9$ & Pa \\
ポアソン比(粒子)$\nu_p$ & 0.23 & - \\
ポアソン比(壁)$\nu_w$ & 0.25 & - \\
摩擦係数(粒子間)$\mu_p$ & 0.0 $\sim$ 1.0 & - \\
摩擦係数(壁-粒子)$\mu_w$ & 0.17 & - \\
容器幅 $W$ & 1.5 & m \\
時間刻み $\Delta t$ & $10^{-7}$ & s \\
\bottomrule
\end{tabular}
\end{table}

\section{シミュレーション結果}
\label{sec:simulation_results}

壁引き抜き法による安息角シミュレーションを実施し、滑り摩擦係数および転がり摩擦係数が安息角に与える影響を系統的に調査した。
粒子数は6670個とし、十分な静止状態に達した後に右壁を引き抜いた。
シミュレーション終了後、斜面表面の粒子座標に対して線形回帰を行い、得られた傾きから安息角を算出した。

\subsection{滑り摩擦係数と安息角の関係}
\label{sec:friction_vs_angle}

滑り摩擦係数 $\mu$ を 0.0 から 1.0 まで変化させ、各条件での安息角を測定した。
転がり摩擦係数は $\mu_r = 0.0$ に固定した。
表\ref{tab:friction_results}に結果を示す。

\begin{table}[htbp]
\centering
\caption{滑り摩擦係数と安息角の関係}
\label{tab:friction_results}
\begin{tabular}{ccc}
\toprule
滑り摩擦係数 $\mu$ [-] & 安息角 $\theta_r$ [deg] & 決定係数 $R^2$ \\
\midrule
0.0 & 0.93 & 0.74 \\
0.2 & 9.64 & 0.92 \\
0.4 & 11.47 & 0.88 \\
0.6 & 12.08 & 0.87 \\
0.8 & 11.80 & 0.89 \\
1.0 & 12.57 & 0.88 \\
\bottomrule
\end{tabular}
\end{table}

図\ref{fig:friction_vs_angle}に、滑り摩擦係数と安息角の関係を示す。
摩擦係数が0のときは約0.9°と非常に低い安息角を示すが、摩擦係数の増加に伴い安息角も増加する傾向が確認できる。
$\mu = 0.2$ で約9.6°、$\mu = 0.4$ 以上では11°〜13°程度で概ね飽和する傾向を示した。
これは、摩擦力が大きくなると粒子間の滑りが抑制され、より急な斜面が安定するためである。

\begin{figure}[htbp]
    \centering
    \includegraphics[width=0.8\textwidth]{figures/friction_vs_repose_angle.png}
    \caption{滑り摩擦係数と安息角の関係。摩擦係数の増加に伴い安息角も増加するが、$\mu > 0.4$ では飽和傾向を示す。}
    \label{fig:friction_vs_angle}
\end{figure}

\subsection{転がり摩擦係数と安息角の関係}
\label{sec:rolling_friction_vs_angle}

次に、転がり摩擦係数 $\mu_r$ の影響を調査した。
滑り摩擦係数を $\mu = 0.5$ に固定し、転がり摩擦係数を 0.0 から 1.0 まで変化させた。
表\ref{tab:rolling_friction_results}に結果を示す。

\begin{table}[htbp]
\centering
\caption{転がり摩擦係数と安息角の関係}
\label{tab:rolling_friction_results}
\begin{tabular}{ccc}
\toprule
転がり摩擦係数 $\mu_r$ [-] & 安息角 $\theta_r$ [deg] & 決定係数 $R^2$ \\
\midrule
0.0 & 1.51 & 0.50 \\
0.2 & 8.24 & 0.92 \\
0.4 & 8.90 & 0.90 \\
0.6 & 9.43 & 0.91 \\
0.8 & 9.51 & 0.93 \\
1.0 & 9.69 & 0.92 \\
\bottomrule
\end{tabular}
\end{table}

図\ref{fig:rolling_friction_vs_angle}に、転がり摩擦係数と安息角の関係を示す。
転がり摩擦係数が 0 の場合、粒子は自由に回転するため、非常に低い安息角(約1.5°)となる。
転がり摩擦係数を増加させると安息角は増加し、$\mu_r = 0.2$ で急激に上昇した後、$\mu_r > 0.4$ では漸増傾向を示す。
これは、転がり摩擦が粒子の回転運動を抑制することで、斜面がより急な角度で安定することを示している。

\begin{figure}[htbp]
    \centering
    \includegraphics[width=0.8\textwidth]{figures/rolling_friction_vs_repose_angle.png}
    \caption{転がり摩擦係数と安息角の関係。転がり摩擦の導入により、安息角が大幅に増加することが確認できる。}
    \label{fig:rolling_friction_vs_angle}
\end{figure}

\section{考察}
\label{sec:discussion}

本シミュレーション結果より、以下のことが考察される。

\begin{enumerate}
    \item \textbf{滑り摩擦の効果}:
    滑り摩擦係数と安息角には正の相関があり、摩擦係数の増加に伴い安息角も増加する傾向が確認された。
    ただし、$\mu > 0.4$ では安息角の増加率が鈍化し、約11°〜13°で飽和する傾向を示した。
    これは、滑り摩擦のみでは斜面を大きく安定化させることに限界があることを示唆している。
    
    \item \textbf{転がり摩擦の重要性}:
    転がり摩擦モデルの導入は、安息角の向上に極めて効果的である。
    転がり摩擦係数 $\mu_r = 0.2$ を導入するだけで、安息角は約1.5°から約8.2°へと大幅に増加した。
    このことから、実際のレゴリス粒子のような非球形粒子の挙動を再現するためには、転がり摩擦モデルが不可欠であることが示唆される。

    \item \textbf{決定係数の解釈}:
    線形回帰の決定係数 $R^2$ は、ほとんどのケースで0.85以上の値を示しており、斜面が概ね直線的に形成されていることが確認できる。
    ただし、摩擦係数が低い場合($\mu = 0.0$ や $\mu_r = 0.0$)では $R^2$ が低下しており、これは粒子が広く拡散して明確な斜面が形成されにくいことを反映していると考えられる。
\end{enumerate}

% ============================================================================
% 第6章:結論
% ============================================================================
\chapter{結論}
\label{chap:conclusion}

\section{本研究のまとめ}
\label{sec:summary}

本研究では、個別要素法(DEM)に基づく粒子シミュレーションプログラムを開発し、その妥当性検証および安息角シミュレーションへの応用を行った。得られた成果を以下にまとめる。

\begin{enumerate}
    \item \textbf{DEMプログラムの開発}\\
    Fortran 90により、Hertz-Mindlin接触力モデル、摩擦モデル、転がり摩擦モデル、クーロン力モデル、セル法による近傍探索を実装した約3000行のDEMプログラムを開発した。
    
    \item \textbf{妥当性の検証}\\
    減衰振動子モデル、壁面接触モデル、斜面モデル、2粒子間クーロン相互作用の4種類の検証計算を実施し、理論解との良好な一致を確認した。これにより、時間積分法および接触力計算の実装が正確であることを実証した。
    
    \item \textbf{安息角シミュレーション}\\
    壁引き抜き法による安息角シミュレーションを実施し、滑り摩擦係数および転がり摩擦係数と安息角の関係を系統的に調査した。
    主な知見として、(1) 滑り摩擦係数の増加に伴い安息角も増加すること、(2) 真球粒子モデルでは理論値より低い安息角となること、(3) 転がり摩擦モデルの導入が安息角の向上に極めて効果的であること、を明らかにした。
    これらの結果は、非球形粒子の挙動を再現するためには転がり摩擦モデルが不可欠であることを示唆している。
\end{enumerate}

\section{今後の課題}
\label{sec:future_work}

本研究の発展として、以下の課題が挙げられる。

\begin{enumerate}
    \item \textbf{3次元への拡張}\\
    現在のプログラムは2次元(平面内運動)に限定されている。3次元への拡張により、より現実的なシミュレーションが可能となる。
    
    \item \textbf{並列計算の導入}\\
    大規模計算に対応するため、OpenMPやMPIによる並列化を行い、計算効率を向上させる。
    
    \item \textbf{非球形粒子への対応}\\
    楕円体や多面体など、非球形粒子の接触力計算を実装することで、より現実的な粒子形状を扱えるようにする。
    
    \item \textbf{実験との比較}\\
    実際の粉体を用いた安息角測定実験を行い、シミュレーション結果との定量的な比較を行う。
    
    \item \textbf{応用展開}\\
    粉体充填、粉体流動、篩分けなど、産業プロセスへの応用を検討する。
\end{enumerate}

% ============================================================================
% 謝辞
% ============================================================================
\chapter*{謝辞}
\addcontentsline{toc}{chapter}{謝辞}

本研究を進めるにあたり、多くの方々にご指導・ご協力をいただきました。

指導教員の○○先生には、研究の方向性から論文の執筆に至るまで、懇切丁寧なご指導を賜りました。ここに深く感謝申し上げます。

また、○○研究室の皆様には、日々の研究活動において多くの助言と励ましをいただきました。心より感謝いたします。

最後に、学生生活を支えてくれた家族に感謝いたします。

\vspace{2cm}
\begin{flushright}
令和X年X月\\
高橋 昇大
\end{flushright}

% ============================================================================
% 参考文献
% ============================================================================
\begin{thebibliography}{99}
\addcontentsline{toc}{chapter}{参考文献}

\bibitem{cundall1979}
P. A. Cundall and O. D. L. Strack,
``A discrete numerical model for granular assemblies,''
\textit{Géotechnique}, Vol. 29, No. 1, pp. 47--65, 1979.

\bibitem{hertz1882}
H. Hertz,
``Über die Berührung fester elastischer Körper,''
\textit{Journal für die reine und angewandte Mathematik}, Vol. 92, pp. 156--171, 1882.

\bibitem{mindlin1953}
R. D. Mindlin and H. Deresiewicz,
``Elastic spheres in contact under varying oblique forces,''
\textit{Journal of Applied Mechanics}, Vol. 20, pp. 327--344, 1953.

\bibitem{thornton1998}
C. Thornton and K. K. Yin,
``Impact of elastic spheres with and without adhesion,''
\textit{Powder Technology}, Vol. 99, pp. 154--162, 1998.

\bibitem{zhou1999}
Y. C. Zhou, B. D. Wright, R. Y. Yang, B. H. Xu, and A. B. Yu,
``Rolling friction in the dynamic simulation of sandpile formation,''
\textit{Physica A}, Vol. 269, pp. 536--553, 1999.

\bibitem{iwashita1998}
K. Iwashita and M. Oda,
``Rolling resistance at contacts in simulation of shear band development by DEM,''
\textit{Journal of Engineering Mechanics}, Vol. 124, pp. 285--292, 1998.

\bibitem{brilliantov1996}
N. V. Brilliantov, F. Spahn, J. M. Hertzsch, and T. Pöschel,
``Model for collisions in granular gases,''
\textit{Physical Review E}, Vol. 53, pp. 5382--5392, 1996.

\bibitem{verlet1967}
L. Verlet,
``Computer `experiments' on classical fluids. I. Thermodynamical properties of Lennard-Jones molecules,''
\textit{Physical Review}, Vol. 159, pp. 98--103, 1967.

\bibitem{allen1987}
M. P. Allen and D. J. Tildesley,
\textit{Computer Simulation of Liquids},
Oxford University Press, 1987.

\bibitem{poschel2005}
T. Pöschel and T. Schwager,
\textit{Computational Granular Dynamics: Models and Algorithms},
Springer, 2005.

\end{thebibliography}

% ============================================================================
% 付録
% ============================================================================
\appendix

\chapter{入力パラメータ一覧}
\label{app:parameters}

表\ref{tab:all_params}に、DEMプログラムで使用可能なすべての入力パラメータを示す。

\begin{table}[htbp]
\centering
\caption{入力パラメータ一覧}
\label{tab:all_params}
\small
\begin{tabularx}{\textwidth}{lXl}
\toprule
パラメータ名 & 説明 & 単位 \\
\midrule
\texttt{TIME\_STEP} & 時間刻み & s \\
\texttt{MAX\_CALCULATION\_STEPS} & 最大計算ステップ数 & - \\
\texttt{YOUNG\_MODULUS\_PARTICLE} & 粒子のヤング率 & Pa \\
\texttt{YOUNG\_MODULUS\_WALL} & 壁のヤング率 & Pa \\
\texttt{POISSON\_RATIO\_PARTICLE} & 粒子のポアソン比 & - \\
\texttt{POISSON\_RATIO\_WALL} & 壁のポアソン比 & - \\
\texttt{PARTICLE\_DENSITY} & 粒子の密度 & kg/m$^3$ \\
\texttt{FRICTION\_COEFF\_PARTICLE} & 粒子間摩擦係数 & - \\
\texttt{FRICTION\_COEFF\_WALL} & 壁-粒子間摩擦係数 & - \\
\texttt{ROLLING\_FRICTION\_COEFF\_PARTICLE} & 粒子間転がり摩擦係数 & - \\
\texttt{ROLLING\_FRICTION\_COEFF\_WALL} & 壁-粒子間転がり摩擦係数 & - \\
\texttt{PARTICLE\_RADIUS\_LARGE} & 大きな粒子の半径 & m \\
\texttt{PARTICLE\_RADIUS\_SMALL} & 小さな粒子の半径 & m \\
\texttt{CONTAINER\_WIDTH} & 容器の幅 & m \\
\texttt{CONTAINER\_HEIGHT} & 容器の高さ & m \\
\texttt{PARTICLE\_GEN\_LAYERS} & 粒子生成層数 & - \\
\texttt{RANDOM\_SEED} & 乱数シード & - \\
\texttt{OUTPUT\_INTERVAL} & 出力間隔 & steps \\
\texttt{ENABLE\_COULOMB\_FORCE} & クーロン力の有効化 & 0/1 \\
\texttt{COULOMB\_CUTOFF} & クーロン力カットオフ半径 & m \\
\texttt{ENABLE\_WALL\_WITHDRAW} & 壁引き抜きの有効化 & 0/1 \\
\texttt{WALL\_WITHDRAW\_STEP} & 壁引き抜き開始ステップ & steps \\
\bottomrule
\end{tabularx}
\end{table}

\chapter{プログラムのフローチャート}
\label{app:flowchart}

図\ref{fig:flowchart}にDEMプログラムの全体フローチャートを示す。

% ここにフローチャートを挿入
% \begin{figure}[htbp]
% \centering
% \includegraphics[width=0.8\textwidth]{figures/flowchart.pdf}
% \caption{DEMプログラムのフローチャート}
% \label{fig:flowchart}
% \end{figure}

\begin{enumerate}
    \item 入力ファイルの読み込み
    \item 粒子の初期配置生成
    \item 材料パラメータの計算
    \item 時間ステップループ開始
    \begin{enumerate}
        \item 近傍粒子探索(セル法)
        \item 接触判定(粒子間、粒子-壁間)
        \item 接触力の計算
        \item クーロン力の計算(有効な場合)
        \item 時間積分(蛙飛び法)
        \item 結果出力(指定間隔ごと)
        \item 静止判定(有効な場合)
    \end{enumerate}
    \item 最終結果の出力
    \item 計算終了
\end{enumerate}

% ============================================================================
% 文書終了
% ============================================================================
\end{document}
